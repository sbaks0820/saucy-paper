Cryptographic protocols are, well, protocols and therefore, follow
a predefined communication pattern.
Our key innovation is to represent such communication protocols using
\emph{session types}.

\paragraph*{\textbf{Example Protocol}}
As an example, consider the two-phase commitment protocol.
The ideal functionality of this protocol consists of a \emph{sender} $S$
and \emph{receiver} $R$ connected to a trusted third-party, which we
name $\fcomm$.
The protocol initiates with $S$ sending a $\mb{commit}$ message to $\fcomm$
indicating its intent to \emph{commit} to a bit.
Next, $S$ sends this committed bit to $\fcomm$.
After receiving the committed bit, $\fcomm$ sends a $\mb{commit}$ message
to $R$ indicating that a bit has been committed to, but does not reveal
this bit to $R$.
At a later time, $S$ sends an $\mb{open}$ message to $\fcomm$ expressing
that $S$ wishes to reveal the secret bit to $R$.
Receiving this message, $\fcomm$ in turn sends an $\mb{open}$ message
to $R$ followed by this bit.
The protocol concludes with each party (process) terminating.

In the session-typed setting, we use typed channels to connect two
parties. For instance, the channel connecting $S$ to $\fcomm$
has the session type $\m{sender}$ defined as
\[
  \mi{stype} \; \m{sender} = \ichoice{\mb{commit} : \m{bit} \product
  \ichoice{\mb{open} : \one}}
\]
The type constructor $\ichoice$ denotes an \emph{internal choice}
(here with only one choice) dictating that $S$ must send the
$\mb{commit}$ message to $\fcomm$.
Next, we use the type constructor $\product$ to denote that $S$
sends a value of type $\m{bit}$ ($\m{bit} \product \ldots$).
We then use the $\ichoiceop$ constructor again enforcing
that $S$ sends $\mb{open}$ to $\fcomm$.
Finally, the type $\one$ denotes termination, indicating that
$S$ will send $\m{close}$ message to $\fcomm$.

Analogously, the channel connecting $R$ and $\fcomm$ has
type $\m{receiver}$ defined as
\[
  \mi{stype} \; \m{receiver} = \echoice{\mb{commit} : 
  \echoice{\mb{open} : \m{bit} \arrow \one}}
\]
Type constructor $\echoice$ represents \emph{external choice}
which is the dual to internal choice.
It prescribes that $R$ must receive a $\mb{commit}$ message from $\fcomm$,
followed by an $\mb{open}$ message (using another $\echoiceop$ constructor).
$R$ must then receive a bit using the $\arrow$ constructor (dual to
$\product$).
Finally, the session terminates as indicated by type $\one$.

Protocols expressed via session types are strictly enforced by
process definitions.
As an illustration, consider the $\fcomm$ process that is connected
to both $S$ and $R$.
The process declaration is written as
\begin{lstlisting}[basicstyle=\small\ttfamily]
decl F_comm :
  (S : sender), (R : receiver) |- (F : 1)
\end{lstlisting}
Here, $\fcomm$ is the name of the process, and $S$ and $R$ are the names
of channels \emph{used} by $\fcomm$, while $F$ is the channel \emph{provided}
by $\fcomm$.
Every session-typed process provides a unique channel while acting as a client
of a non-negative number of channels.
The used channels with their types are written to the left of the turnstile
($\vdash$) while the offered channel and type are written on the right.
This is analogous to function definitions where used channels correspond to
arguments, while offered channel corresponds to the result.

The $\fcomm$ process is defined as follows:
\begin{lstlisting}[basicstyle=\small\ttfamily, numbers=left,xleftmargin=2em]
proc F <- F_comm S R =
  case S (
    commit => b = recv S ;
              R.commit ;
              case S (
                open => R.open ;
                        send R b ;
                        wait S ; wait R ;
                        close F ) )
\end{lstlisting}
The process first case analyzes on channel $S$ branching on the
message received.
Since there is only one choice $\mb{commit}$, we only have one
branch in the definition.
$\fcomm$ then receives the bit $b$ (line 3) on $S$, followed by sending the
commit message on channel $R$ (line 4).
Once $\fcomm$ receives the $\mb{open}$ message on $S$, it sends the
$\mb{open}$ message on $R$ (line 6), followed by the bit $b$ (line 7).
Finally, the process waits for channels $S$ and $R$ to close (line 8),
followed by ultimately closing the channel $F$ (line 9).


\begin{figure*}[!ht]
  In this section we use our emulation definition to model a simple commitment protocol in the random oracle mode.
We present a complete emulation from the canonical $\F_{\msf{com}}$ definition with a simulator for the dummy adversary.

The ideal functionality \Fcom is the same one introduced in Figure~\ref{fig:fcom} in Section~\ref{sec:nomosuc}. 
In FIgure~\ref{lst:fcom} we present the Nomos definition of the same functionality and provide the channel types in FIgure~\ref{fig:fcomtypes}.
We elide some of the clutter of acquiring and releasing shared channels in the form of: \texttt{\$p2f $\leftarrow$ acquire \#p\_to\_f} for clarity. 
Wherever a linear channel like \texttt{\$p2f} is used it is in fact an acquired shared channel \texttt{\#p\_to\_f}.

The key difference to note between \Fcom and its Nomos version is that the functionality is split up into two processes rather than compressed into one.
This design decision is required because of how Nomos cycles between processes in a round-robin fashion and the communicator design.
Therefore, processes must recurse when there is no message to be read and move to the next processes after the first expected message is received--in this case the \msf{P2FCommit(b)} message.

\begin{figure}
\centering
\msf{type} \msf{Ip2f} = \msf{P2FCommit} of \msf{Bit} | \msf{P2FOpen}

\msf{type} \msf{If2p} = \msf{F2PCommit} | \msf{F2POpen} of \msf{Bit}

\msf{type} \msf{Ip2f} = \msf{SCommit} of Bit | \msf{SOpen}

\msf{type} \msf{If2p} = \msf{RCommit} | \msf{ROpen} of Bit

\msf{type} \msf{Rp2f} = \msf{SHash} of \msf{Int} | \msf{Send} of \msf{pid} \textasciicircum \msf{pid} \textasciicircum \msf{Int}

\msf{type} \msf{Rf2p} = \msf{Pre} of \msf{Int} | \msf{RHash} of \msf{Int} | \msf{MSG} of \msf{pid} \textasciicircum \msf{pid} \textasciicircum \msf{Int}

\caption{Types for the channels in the ideal world for \Fcom. Notice that Ip2f and If2p is the type of the channels \msf{z2p} and \msf{p2z} as they much match for both worlds and the ideal world parties simply forward messages to the functionality. The \msf{p2f} and \msf{f2p} channels are specific to the real and ideal world as the functionalities are not the same. Hence the real-world \msf{p2f} is typed with \msf{Rp2f} for the random oracle and the ideal world \msf{p2f} is typed with \msf{Ip2f} for \Fcom.}
\label{fig:fcomtypes}
\end{figure}

\begin{figure*}
\begin{lstlisting}[basicstyle=\small\BeraMonottFamily]
proc F_code:
  (s: sid), (k: Int), (rng: [Bit]), (clist: list[Int]),
  (#p_to_f: comm[pid ^ Ip2f]{Ip2fn}), (#f_to_p: comm[pid ^ If2p]{If2pn}),
  (#a_to_f: comm[Ia2f]{Ia2fn}), (#f_to_a: comm[If2a]{If2an})  |- ($ch: FtOE) =
{
  case $p2f (
    yes =>	
      pid, msg = recv $p2f ;
      get $pwf {Ip2fn} ;
      case msg (
        P2FCommit(b) =>	
          if pid == 1
          then
            send F2PCommit $f2p ;
            pay {If2pn} $f2p ;
            $ch <- F_com_open s k rng clist #p_to_f #f_to_p b ;
          end
      )
   | no =>  
       $ch <- F_code s k rng clist #p_to_f #f_to_p ;
  )
}

proc F_code_open:
{
  case $p2f (
    yes =>	
      pid, msg = recv $p2f ;
      get $p2f {0} ;
      case msg (
        P2FOpen =>	
          if pid == 1
          then
            send F2POpen(b) $f2p ;
            pay {0} $f2p ;
            $ch <- 1 ;
          end
      )
   | no =>
       $ch <- F_code_open s k rng clist #p_to_f #f_to_p b ;
  )
}
\end{lstlisting}
\end{figure*}

The real world protocol for commitment follows a simple communication patter:
\begin{enumerate}
\item On input bit $(\msf{P2FCommit}\ b)$ from the environment, the committer queries the random oracle with the message $(\msf{SHash}\ b | r)$ where $r \xleftarrow{\$} \{0,1\}^k$.
The returned ``hash value'' is sent to the receiver as the commitment.
\item On input \msf{P2FOpen} from \Environment, the committer sends $(\msf{Send}\ p_L\ b\ r)$ to the receiver.
\item The receiver checks that the commitment is correct be querying \Fro in the same way and asserting that the has returned $(\msf{RHash}\ h)$ is the same as the one sent by $p_C$.
\item \todo{the type of Rf2p is kind of wrong so need to correct it}
\end{enumerate}

We provide only a simulator for the dummy adversary as that guarantees a simulator for all adversaries required our emulation definition.
The simulator for commitment is relatively simple so we only provide a high-level description here and leave the full simulator code to the appendix.
The simulator internally simulats the random oracle by maintaining a table of key-value pairs that it can control entirely.
If the committer is corrupt:
\begin{enumerate}
\item The simulator can not determine the bit \Environment wants to commit to so selects a random bit when activatd by the environment and gives it as input to the corrupted committer.
\item When it's asked to open the commitment it simply forwards the request to the corrupt committer and stops.
\end{enumerate}
If the receiver is corrupt:
\begin{enumerate}
\item When activated by the receiver with (\msf{F2PCommit}), the simulator generates some random string $h$ to represent the commitment, stores it, and sends ($\msf{P2A}\ \msf{MSG}(p_C, p_R, h)$) to \Environment.
\item When it receives ($\msf{F2POpen}\ b$) from the receiver, it returns $(\msf{P2A}\ \msf{MSG}(p_C, p_R, b, r)$ to \Environment where $r$ is a randomly generated sequence keeping the pair $(b | r, h)$ as the corresponding entry in the table.
\item When activated by \Environment to check the commitment, \Simulator simply returns the commitment hash or creates a new one.
\end{enumerate}

\paragraph{Simulator Well-Matched}
It is immediately obvious that the constructed simulator is well-typed if the dummy adversary is well-typed with the given type parameters.
The simulator receives 1 import token per activation from \Environment which suffices to simulated \Fro internally. 
Subsequently, \Simulator keeps all of the import it receives, and, therefore when one of the partiesis corrupt a simple bounding polynomial can be given as:
\[
	T_{\Dummysim}(n) = T_{\Fro}(n) + O(1)
\]
where $T_{\Fro}$ is a satisfying polynomial for \Fro. The additional constant factor simply accounts for sending messages to the corrupt parties.
Therefore,
\begin{gather}
	\forall \Environment, \langle \Environment \leftrightarrow \DummyAdv \rangle \Rightarrow \langle \Environment \leftrightarrow \Dummysim \rangle
\end{gather}


\begin{figure*}
\begin{lstlisting}[basicstyle=\BeraMonottFamily]
(* Z2P interface *)
type Ip2f = P2FCommit of Bit | P2FOpen ;
type If2p = F2PCommit | F2POpen of Bit ;

(* Ideal World *)
type Ip2f = SCommit of Bit | SOpen ;
type If2p = RCommit | ROpen of Bit ;

type Ia2f = 0 ;
type If2a = 0 ;

type Ia2p = Ip2f ;	(* crupt input is same as z2p *)
type Ip2a = If2p ;

(* Real World *)
type Rp2f = SHash of Int | Send of pid ^ pid ^ Int ;
type Rf2p = Pre of Int | RHash of Int | MSG of pid ^ pid ^ Int ;

type Ra2f = A2Hash of Int ;
type Rf2a = Hash2A of Int ;

type Ra2p = Rp2f ;
type Rp2a = Rf2p ;

(* the import here is given as those for the dummy adversary in the real world *)
p2zn <- 0 ; z2pn <- 1 ;
a2zn <- 0 ; z2an <- 1 ; 

Rf2pn <- 0 ; Rp2fn <- 1 ;
Rp2an <- 0 ; Ra2pn <- 1 ;
Rf2an <- 0 ; Ra2fn <- 1 ;

If2pn <- 0 ; Ip2fn <- 0 ;
Ip2an <- 0 ; Ia2pn <- 0 ;
I

(* channels *)
#z_to_p <- comm[pid ^ Ip2f]
#p_to_z <- comm[pid ^ If2p]
#z_to_a <- comm[ z2d[Ra2p][Ra2f] ] ;
#z_to_z <- comm[ d2z[Rp2a][Rf2a] ] ;


(* Real World exec PI *)
execUC[Ip2f][If2p][Rp2f][Rf2p][Rp2a][Ra2p][Rf2a][Ra2f][a2z][z2a]
	  {p2zn}{z2pn}{f2pn}{p2fn}{p2an}{a2pn}{f2an}{a2fn}{a2zn}{z2an}

(* Ideal world exec PHI *)
execUC[If2p][Ip2f][Ip2f][If2p][Ip2a][Ia2p][If2a][Ia2f][a2z][z2a]
	  {p2zn}{z2pn}






\end{lstlisting}
\end{figure*}

  \caption{The $\mathcal{F}_{\msf{comm}}$
  commitment ideal functionality in Nomos.
  The types for the sender and receiver channel define what inputs they
  can give to the functionality and what messsages are sent from the
  functionality back to the receiver.}
  \label{fig:nomos:commitment}
  \end{figure*}

\subsection{Formal Description of Nomos}
Nomos internally relies on session types to express and enforce protocols
of interaction between different parties.
Session types are a type discipline for communication-centric programming
based on message passing via channels.
The underlying base system of session types is derived from a Curry-Howard
interpretation~\cite{Caires10concur,Caires16mscs} of intuitionistic linear logic
\cite{Girard87tapsoft}. The key idea is that an intuitionistic linear sequent
$A_1, A_2, \ldots, A_n \vdash A$
is interpreted as the interface to a process expression $P$. We label each of the
antecedents with a channel name $x_i$ and the succedent with channel name $z$.
The $x_i$'s are \emph{channels used by} $P$ and $z$ is the \emph{channel provided by} $P$.
\begin{center}
  \begin{minipage}{0cm}
  \begin{tabbing}
  $x_1 : A_1, x_2 : A_2, \ldots, x_n : A_n \vdash P :: (z : C)$
  \end{tabbing}
  \end{minipage}
\end{center}
The resulting judgment formally states that process $P$ provides a service of
session type $C$ along channel $z$, while using the services of session types $A_1,
\ldots,A_n$ provided along channels $x_1, \ldots, x_n$ respectively.
We abbreviate the antecedent of the sequent by $\Delta$.

The standard connectives of intuitionistic linear logic give rise
to the type constructors in session types.
In this work, we restrict ourselves 
In addition to the type constructors arising
from the connectives of intuitionistic linear logic, we have type names,
indexed by a
sequence of arithmetic expressions $V \indv{e}$, existential and
universal quantification over natural numbers ($\texists{n} A$,
$\tforall{n} A$) and existential and universal constraints
($\tassert{\phi} A$, $\tassume{\phi} A$).  We write $i$ for constant
and $n$ for variable natural numbers.
\[
  \begin{array}{lrcl}
    \mbox{Types} & A, B & ::= & \ichoice{\ell : A_\ell}_{\ell \in L}
    \mid \echoice{\ell : A_\ell}_{\ell \in L} \\
                 & & \mid & A \tensor B \mid A \lolli B \mid \one \mid V \indv{e} \\
                 & & \mid & \tassert{\phi} A \mid \tassume{\phi} A
                            \mid \texists{n} A \mid \tforall{n} A \\[0.5em]
    \mbox{Arith. Exps.} & e & ::= & i \mid e + e \mid e - e \mid i \times e \mid n \\[0.5em]
    \mbox{Arith. Props.} &
    \phi & ::= & e = e \mid e > e \mid \top \mid \bot
                 \mid \phi \land \phi \\
    & & \mid &  \phi \lor \phi \mid \lnot \phi \mid \texists{n}\phi \mid \tforall{n} \phi \\[0.5em]
    \mbox{Procs} & P, Q & ::= & \esendl{x}{k} \semi P \mid \ecase{x}{l}{P}_{l \in L} \\
                & & \mid & \esendch{x}{y} \semi P \mid \erecvch{x}{y} \semi P \\
                & & \mid & \eclose{x} \mid \ewait{x} \semi P \\
                & & \mid & \fwd{x}{y} \mid \ecut{x}{f}{y}{P} \\
                & & \mid & \eassert{x}{\phi} \semi P \mid \eassume{x}{\phi} {\semi} P
  \end{array}
\]
Our implementation does not support type polymorphism but it is convenient in
some of the examples.  We therefore allow definitions such as
$\queue{A}[n] = \ldots$ and interpret them as a family of definitions,
one for each possible type $A$.

The typing judgment has the form of a sequent
\begin{center}
  \begin{minipage}{0cm}
  \begin{tabbing}
  $\vars \semi \cons \semi \D \vdash_\Sg P :: (x : A)$
  \end{tabbing}
  \end{minipage}
\end{center}
where $\vars$ are index variables $n$, $\cons$ are constraints over
these variables expressed as a single proposition,
% $\psi$\footnote{\it should we adopt maybe $\psi$ for the constraints?}
$\D$ are the linear antecedents $x_i : A_i$, $P$ is a process
expression, and $x : A$ is the linear succedent. We propose and maintain
that the $x_i$'s and $x$ are all distinct, and that all free index
variables in $\cons$, $\D$, $P$, and $A$ are contained among $\vars$.
Finally, $\Sigma$ is a fixed signature containing type and process
definitions (explained in Section~\ref{subsec:base}) Because it is
fixed, we elide it from the presentation of the rules.  In addition we
write $\vars \semi \cons \proves \phi$ for semantic entailment
(proving $\phi$ assuming $\cons$) in the
constraint domain where $\vars$ contains all arithmetic variables in
$\cons$ and $\phi$.
Table~\ref{tab:language} overviews the session types their associated
process terms, their continuation (both in types and terms) and operational description.
% Figure~\ref{fig:basic-typing} describes selected
% typing rules (ignore the premises and annotation on the turnstile
% marked in blue, introduced in Section~\ref{subsec:ergo}) leaving the complete
% set of rules to Appendix~\ref{app:formal}.

We formalize the operational semantics as a system of \emph{multiset rewriting
rules}~\cite{Cervesato09SEM}. We introduce semantic objects $\proc{c}{P}$
and $\msg{c}{M}$ which mean that process $P$ or message $M$ provide
along channel $c$.
A process configuration is a multiset of such objects, where any two
channels provided are distinct
(formally described in Section~\ref{subsec:soundness}).

\subsection{Basic Session Types}\label{subsec:base}
In this subsection, we review the syntax and semantics for the basic
session type operators ($\with$, $\oplus$, $\tensor$, $\lolli$ and $\one$).
A summary of the corresponding process terms and intuitive explanation
for semantics is provided in Table~\ref{tab:language}.

\paragraph{\textbf{External Choice}}
The \emph{external choice} type constructor
$\echoice{\ell : A_{\ell}}_{\ell \in L}$ is an $n$-ary labeled
generalization of the additive conjunction $A \with B$. Operationally,
it requires the provider of
$x : \echoice{\ell : A_{\ell}}_{\ell \in L}$ to branch based on the
label $k \in L$ it receives from the client and continue to provide
type $A_{k}$. The corresponding process term is written as
$\ecase{x}{\ell}{P}_{\ell \in L}$.  Dually, the client must send one
of the labels $k \in L$ using the process term
$(\esendl{x}{k} \semi Q)$ where $Q$ is the continuation.
\begin{mathpar}
\infer[{\with}R]
{\vars \semi \cons \semi \D \vdash \ecase{x}{\ell}{P_\ell}_{\ell \in L} ::
(x : \echoice{\ell : A_\ell}_{\ell \in L})}
{(\forall \ell \in L)
 & \vars \semi \cons \semi \D \vdash P_\ell :: (x : A_\ell)}
\and
\infer[{\with}L]
{\vars \semi \cons \semi \D, (x : \echoice{\ell : A_\ell}_{\ell \in L}) \vdash
(\esendl{x}{k} \semi Q) :: (z : C)}
{(k \in L) & \vars \semi \cons \semi \D, (x : A_k) \vdash Q :: (z : C)}
\end{mathpar}
Communication is asynchronous, so that the client
$\esendl{c}{k} \semi Q$ sends a message $k$ along $c$ and continues as $Q$
without waiting for it to be received. As a technical device to ensure that
consecutive messages on a channel arrive in order, the sender also creates a
fresh continuation channel $c'$ so that the message $k$ is actually represented
as $(\esendl{c}{k} \semi \fwd{c}{c'})$ (read: send $k$ along $c$ and continue along
$c'$). When the message $k$ is received along $c$, we select branch $k$ and
also substitute the continuation channel $c'$ for $c$.
Rules ${\with}S$ and ${\with}C$ below describe the operational behavior of the
provider and client respectively $\fresh{c'}$.

\begin{tabbing}
$({\with}S) : \m{proc}(d, c.k \semi Q) \;\mapsto\; \m{msg}(c', c.k \semi c' \leftarrow c),
\m{proc}(d, Q[c'/c])$ \\
$({\with}C):$ \= $\m{proc}(c, \m{case}\;c\;(\ell \Rightarrow Q_\ell)_{\ell \in L}),$\\
\> $\m{msg}(c', c.k \semi c' \leftarrow c)
\;\mapsto\; \m{proc}(c', Q_k[c'/c])$
\end{tabbing}

The \emph{internal choice} constructor
$\ichoice{\ell : A_{\ell}}_{\ell \in L}$ is the dual of external
choice requiring the provider to send one of the labels $k \in L$ that
the client must branch on.
\begin{mathpar}
  \infer[{\oplus}R]
    {\vars \semi \cons \semi \D \vdash (\esendl{x}{k} \semi P) :: (x : \ichoice{\ell : A_\ell}_{\ell \in L})}
    {(k \in L) & \vars \semi \cons \semi \D \vdash P :: (x : A_k)}
  \and
  \infer[{\oplus}L]
    {\vars \semi \cons \semi \D, (x : \ichoice{\ell : A_\ell}_{\ell \in L}) \vdash
    \ecase{x}{\ell}{Q_\ell}_{\ell \in L} :: (z : C)}
    {(\forall \ell \in L) &
      \vars \semi \cons \semi \D, (x : A_\ell) \vdash Q_\ell :: (z : C)}
\end{mathpar}
This dual constructor reverses the role of the provider and client.
The provider $(\esendl{x}{k} \semi P)$ of $x : \ichoice{\ell : A_\ell}_{\ell \in L})$
sends the label $k$ along $x$ and continues to provide $x : A_k$.
Correspondingly, the client branches on the label received using channel
$x : A_\ell$ in branch $\ell$ with process term $Q_\ell$.
The rules of operational semantics (${\oplus}S, {\oplus}C$) are exact dual
of ${\with}S$ and ${\with}C$ and omitted for brevity.

\paragraph{\textbf{Channel Passing}}
The \emph{tensor} operator $A \tensor B$ prescribes that the provider of
$x : A \tensor B$
sends a channel $y$ of type $A$ and continues to provide type $B$. The
corresponding process term is $\esendch{x}{y} \semi P$ where $P$ is
the continuation.  Correspondingly, its client must receives a channel
using the term $\erecvch{x}{y} \semi Q$, binding it to variable $y$
and continuing to execute $Q$.
\begin{mathpar}
  \infer[{\tensor}R]
    {\vars \semi \cons \semi \D, (y : A) \vdash (\esendch{x}{y} \semi P) :: (x : A \tensor B)}
    {\vars \semi \cons \semi \D \vdash P :: (x : B)}
  \and
  \infer[{\tensor}L]
    {\vars \semi \cons \semi \D, (x : A \tensor B) \vdash (\erecvch{x}{y} \semi Q) :: (z : C)}
    {\vars \semi \cons \semi \D, (y : A), (x : B) \vdash Q :: (z : C)}
\end{mathpar}
Operationally, the provider $\esendch{c}{d} \semi P$ sends the
channel $d$ and the continuation channel $c'$ along $c$ as a message and
continues with executing $P$. The client receives the channel $d$ and continuation
channel $c'$ and substitutes $d$ for $x$ and $c'$ for $c$.
\begin{tabbing}
$({\tensor}S) : \m{proc}(c, \m{send}\; c\; d \semi P) \;\mapsto\; $\=
$\m{proc}(c', P[c'/c]),$\\
\>$\m{msg}(c, \m{send}\; c\; d \semi c \leftarrow c')$ \\
$({\tensor}C) : $ \= $\m{msg}(c, \m{send}\; c\; d \semi c \leftarrow c'),$\\
\>$\m{proc}(e, x \leftarrow \m{recv}\; c \semi Q)
\;\mapsto\; \m{proc}(e, Q[c', d/c, x])$

\end{tabbing}
The dual operator $A \lolli B$ allows the provider to receive a
channel of type $A$ and continue to provide type $B$. The client
of $A \lolli B$, on the other hand, sends the channel of type $A$
and continues to use $B$.
\begin{mathpar}
  \infer[{\lolli}R]
    {\vars \semi \cons \semi \D \vdash (\erecvch{x}{y} \semi P) :: (x : A \lolli B)}
    {\vars \semi \cons \semi \D, (y : A) \vdash P :: (x : B)}
  \and
  \infer[{\lolli}L]
    {\vars \semi \cons \semi \D, (x : A \lolli B), (y : A) \vdash (\esendch{x}{y} \semi Q) :: (z : C)}
    {\vars \semi \cons \semi \D, (x : B) \vdash Q :: (z : C)}
\end{mathpar}

\paragraph{\textbf{Termination}}
The type $\one$, the multiplicative unit of linear logic,
indicates \emph{termination} requiring that the provider send a
\emph{close} message followed by terminating the communication.
Linearity enforces that the provider not use any channels.
\begin{mathpar}
  \infer[{\one}R]
    {\vars \semi \cons \semi \cdot \vdash (\eclose{x}) :: (x : \one)}
    {}
  \and
  \infer[{\one}L]
    {\vars \semi \cons \semi \D, (x : \one) \vdash (\ewait{x} \semi Q) :: (z : C)}
    {\vars \semi \cons \semi \D \vdash Q :: (z : C)}
\end{mathpar}
Operationally, the provider waits for the closing message, which
has no continuation channel since the provider terminates.
\begin{tabbing}
$({\one}S) : \m{proc}(c, \m{close}\; c) \;\mapsto\; \m{msg}(c, \m{close}\; c)$ \\
$({\one}C) : $ \= $\m{msg}(c, \m{close}\; c),
\m{proc}(d, \m{wait}\; c \semi Q) \;\mapsto\; \m{proc}(d, Q)$
\end{tabbing}

\paragraph{\textbf{Forwarding}}
A process $\fwd{x}{y}$ identifies the channels $x$ and $y$ so that any
further communication along either $x$ or $y$ will be along the unified
channel. Its typing rule corresponds to the logical rule of identity.
\begin{mathpar}
  \infer[\m{id}]
    {\vars \semi \cons \semi y : A \vdash (\fwd{x}{y}) :: (x : A)}
    {}
\end{mathpar}
Operationally, a process $\fwd{c}{d}$ \emph{forwards} any message M
that arrives on $d$ to $c$ and vice-versa. Since channels are used
linearly, the forwarding process can then terminate, ensuring proper
renaming, as exemplified in the rules below.
\begin{tabbing}
$(\m{id}^+C) : $ \= $\m{msg}(d, M),
\m{proc}(c, c \leftarrow d) \;\mapsto\; \m{msg}(c, [c/d]M)$ \\
$(\m{id}^-C) : $ \= $\m{proc}(c, c \leftarrow d),
\m{msg}(e, M(c)) \;\mapsto\; \m{msg}(e, [d/c]M(c))$
\end{tabbing}
We write $M(c)$ to indicate that $c$ must occur in message $M$ ensuring that $M$ is the
sole client of $c$.

\paragraph{\textbf{Process Definitions}}
Process definitions have the form
$\D \vdash f \indv{n} = P :: (x : A)$ where $f$ is the name of the
process and $P$ its definition. In addition, $\overline{n}$ is a
sequence of arithmetic variables that $\D$, $P$ and $A$ can refer to.
All definitions are collected in a fixed global signature $\Sg$.
For a \emph{well-formed signature}, we
require that $\overline{n} \semi \top \semi \D \vdash P :: (x : A)$
for every definition, thereby allowing
definitions to be mutually recursive. A new instance of a defined
process $f$ can be spawned with the expression
$\ecut{x}{f \indv{e}}{\overline{y}}{Q}$ where $\overline{y}$ is a
sequence of channels matching the antecedents $\D$ and $\indv{e}$ is a
sequence of arithmetic expression matching the variables
$\indv{n}$. The newly spawned process will use all variables in
$\overline{y}$ and provide $x$ to the continuation $Q$.
\begin{mathpar}
  \inferrule*[right=$\m{def}$]
  {\overline{y':B} \vdash f \indv{n} = P_f :: (x' : A) \in \Sg \\
  \D' = \overline{(y:B)}[\overline{e}/\overline{n}] \and
  \vars {\semi} \cons {\semi} \D, (x : A[\overline{e}/\overline{n}]) \vdash Q :: (z : C)}
  {\vars \semi \cons \semi \D, \D' \vdash (\ecut{x}{f \indv{e}}{\overline{y}}{Q}) :: (z : C)}
\end{mathpar}
The declaration of $f$ is looked up in the signature $\Sg$ (first premise), and $\overline{e}$
is substituted for $\overline{n}$ while matching the types in $\D'$ and $\overline{y}$
(second premise). Similarly, the freshly created channel $x$ has type $A$ from the signature
with $\overline{e}$ substituted for $\overline{n}$.
The corresponding semantics rule also performs a similar substitution
$\fresh{a}$.
\begin{tabbing}
$(\m{def}C) : $ \= $\m{proc}(c, \ecut{x}{f \indv{e}}{\overline{d}}{Q}) \; \mapsto \;$ \\
\> $\m{proc}(a, P_f[a/x, \overline{d}/\overline{y'}, \overline{e}/\overline{n}]), \;
   \m{proc}(c, Q[a/x])$
\end{tabbing}
where $\overline{y' : B} \vdash f \indv{n} = P_f :: (x' : A) \in \Sg$.

Sometimes a process invocation is a tail call,
written without a continuation as $\procdef{f \indv{e}}{\overline{y}}{x}$. This is a
short-hand for $\procdef{f \indv{e}}{\overline{y}}{x'} \semi \fwd{x}{x'}$ for a fresh
variable $x'$, that is, we create a fresh channel
and immediately identify it with x.

\paragraph{\textbf{Type Definitions}}
As our queue example already showed, session types can be defined
recursively, departing from a strict Curry-Howard interpretation of
linear logic, analogous to the way pure ML or Haskell depart from
a pure interpretation of intuitionistic logic.  For this purpose we
allow (possibly mutually recursive) type definitions $V \indv{n} = A$
in the signature $\Sg$. Here, $\overline{n}$ denotes a sequence of
arithmetic variables. Again, for a well-formed signature, we require $A$ to be
\emph{contractive}~\cite{Gay2005} meaning $A$ should not itself be a
type name. Our type definitions are \emph{equirecursive} so we can
silently replace type names $V \indv{e}$ indexed with arithmetic
refinements by $A [\overline{e}/\overline{n}]$ during type checking,
and no explicit rules for recursive types are needed.

All types in a signature must be \emph{valid}, formally denoted with the judgment
$\vars \semi \cons \vdash A\; \mi{valid}$, which requires that all
free arithmetic variables of $\cons$ and $A$ are contained in $\vars$,
and that for each arithmetic expression $e$ in $A$ we can prove
$\vars' \semi \cons' \vdash e : \m{nat}$ for the constraints $\cons'$
known at the occurrence of $e$ (implicitly proving that $e \geq 0$).


\subsection{Preservation and Progress}
The main theorems that exhibit the deep connection between our type
system and the operational semantics are the usual \emph{type
  preservation} and \emph{progress}, sometimes called \emph{session
  fidelity} and \emph{deadlock freedom}, respectively.

So far, I have only described individual processes. However, processes
exist in a \emph{configuration}. A process configuration is a multiset
of semantic objects, $\proc{c}{P}$ and $\msg{c}{M}$, where any
two offered channels are distinct. A key question is how to type these
configurations. Since they consist of both processes and messages, they
both \emph{use} and \emph{provide} a collection of channels.
And even though a configuration is treated as a multiset, typing imposes
a partial order on the processes and messages where a provider of a
channel appears to the left of its client.

A configuration is typed w.r.t. a signature providing the type declaration
of each process.
A signature $\Sg$ is \emph{well formed} if
(a) every type definition $V = A_V$ is \emph{contractive},
and (b) every process definition
$\D \vdash f = P :: (x : A)$ in $\Sg$
is well typed according to the process typing judgment, i.e.
$\Sg \semi \D \vdash P :: (x : A)$.

I use the following judgment to type a configuration.
\[
\Sg \semi \D_1 \vDash \config :: \D_2
\]
It states that $\Sg$ is well-formed
and that the configuration $\config$
uses the channels in the context $\D_1$ and provides
the channels in the context $\D_2$.
\begin{figure}[t]
\begin{mathpar}
\infer[\m{empty}]
{\Sg \semi \D \vDash (\cdot) :: \D}
{}
\and
\infer[\m{compose}]
{\Sg \semi \D_0 \vDash (\config_1 \; \config_2) :: \D_2}
{\Sg \semi \D_0 \vDash \config_1 :: \D_1 \qquad
\Sg \semi \D_1 \vDash \config_2 :: \D_2}
\and
\infer[\m{proc}]
{\Sg \semi \D, \D_1 \vDash \proc{c}{P} :: (\D, (c : A) )}
{\Sg \semi \D_1 \vdash P :: (c : A)}
\and
\infer[\m{msg}]
{\Sg \semi \D, \D_1 \vDash \msg{c}{P} :: (\D, (c : A) )}
{\Sg \semi \D_1 \vDash P :: (c : A)}
\end{mathpar}
\caption{Typing rules for a configuration}
\label{fig:config_typing}
\end{figure}
The configuration typing judgment is defined using
the rules presented in Figure~\ref{fig:config_typing}.
%
The rule $\m{empty}$ defines that an empty configuration
is well-typed. The rule $\m{compose}$
composes two
configurations $\config_1$ and $\config_2$; $\config_1$ provides
service on the channels in $\D_1$ while $\config_2$ uses
the channels in $\D_2$. The $\m{proc}$ rule creates a configuration
out of a single process. Similarly, the $\m{msg}$ rule creates a
configuration out of a single message.

\begin{theorem}[Type Preservation]
\label{thm:preservation}
If $\Sg \semi \D' \vDash \config :: \D$ and $\config \step \dc$,
then $\Sg \semi \D' \vDash \dc :: \D$.
\end{theorem}
\begin{proof}
  By case analysis on the transition rule, applying inversion to the
  given typing derivation, and then assembling a new derivation of
  $\dc$.
\end{proof}

A process or message is said to be \emph{poised} if it is trying to
communicate along the channel that it provides.  A poised process is
comparable to a value in a sequential language. A configuration is
poised if every process or message in the configuration is poised.
Conceptually, this implies that the configuration is trying to communicate
externally, i.e. along one of the channel it provides.
The progress theorem then shows that either a configuration can take a
step or it is poised.  To prove this I show first that the typing
derivation can be rearranged to go strictly from right to left and
then proceed by induction over this particular derivation.

\begin{theorem}[Global Progress]
\label{thm:progress}
\mbox{}
If $\cdot \vDash \config :: \D$ then either
\begin{enumerate}
\item[(i)] $\config \mapsto \dc$ for some $\dc$, or
\item[(ii)] $\config$ is poised.
\end{enumerate}
\end{theorem}
\begin{proof}
By induction on the right-to-left typing of $\config$ so that either
$\config$ is empty (and therefore poised) or
$\config = (\dc\; \proc{c}{P})$ or
$\config = (\dc\; \msg{c}{M})$. By induction hypothesis, $\dc$ can
either take a step (and then so can $\config$), or $\dc$ is poised.  In
the latter case, I
analyze the cases for $P$ and $M$, applying multiple steps of
inversion to show that in each
case either $\config$ can take a step or is poised.
\end{proof}


\subsection{UC Communicators}
The linear aspect of session types imposes an important restriction on programs.
The only provision to spawn new processes is when a parent process creates a new
child process, and uses an exclusive linear channel to communicate with the child.
Thus, any two processes connected by a channel inherently maintain this parent-child
relationship.
Intuitively, this leads to a linear tree-like hierarchy among the processes,
thus preventing a cycle in the process graph.

Unfortunately, this restriction precludes practical programming scenarios
where process topologies indeed have a cyclic dependency (e.g. ring networks,
dining philosophers, etc.).
Recognizing this limitation, Balzer et al.~\cite{Balzer17icfp} proposed
a \emph{shared} extension of session types that allows arbitrary process topologies.
We have found this extension exceedingly helpful in the design and implementation
of cryptographic protocols.

One illustration of such a use of shared session types is a \emph{communicator}.
We use communicators as message buffers between two arbitrary processes: a
\emph{sender} and a \emph{receiver}.
The communicator is connected to both the sender and the receiver using a shared
channel.
Intuitively, the communicator receives \emph{push} requests from the sender followed
by receiving a message and stores them internally.
Analogously, the communicator receives \emph{pop} requests from the receiver,
and responds appropriately with the message if one is stored inside the communicator.
Formally, a communicator has the following polymorphic session type
\begin{tabbing}
  $\mi{stype} \; \m{comm[msg]} =$\\
  \quad $\up \echoice{$\=$\mb{push} : \m{msg} \arrow
  \down \m{comm[msg]},$\\
  \>$\mb{pop} : \ichoice{$\=$\mb{yesmsg} : \m{msg} \product \down \m{comm[msg]},$\\
  \>\>$\mb{nomsg} : \down \m{comm[msg]} }}$
\end{tabbing}
The type $\m{comm}$ is parameterized by the type $\m{msg}$, i.e., the type of
messages in the buffer.
The type initiates with an $\up$ denoting that $\m{comm}$ is a shared session type.
The type prescribes that the communicator needs to be acquired by the sender (or receiver)
for further interaction.
Such an acquire-release discipline is automatically enforced by the shared session type.
Once acquired, the communicator can either receive $\mb{push}$ (from sender) or
$\mb{pop}$ requests (from receiver).
In the former case, the communicator receives a message of type $\m{msg}$, and
then detaches from the client using the dual $\down$ operator.
In the latter case, the communicator checks if it internally contains a message
for the receiver.
If yes, the communicator replies with the $\mb{yesmsg}$ label followed by sending
the message (the $\product$ constructor).
Otherwise, the communicator replies with the $\mb{nomsg}$ label.
In either case, the communicator then detaches from the client matching the $\down$
operator.
Internally, the communicator stores these messages in a first-in-first-out order.

The communicator is also the perfect opportunity to implement an unreliable
message buffer that can drop or reorder messages.
All we would need to do is change the internal implementation of the communicator
\emph{without} changing the offered session type.
