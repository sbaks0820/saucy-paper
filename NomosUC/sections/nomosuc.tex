% Comments:
% Why use session types? The advantages
% Maybe have an untyped commitment protocol
% Benefit of session types: extra type annotations, concise specification
% Does not provide a complete term, only a specification
% Might make sense to talk about extending session types with dependencies for commitment

Cryptographic protocols are, well, protocols and therefore, follow
a predefined communication pattern.
Our key innovation is to represent such communication protocols using
\emph{session types}.

\paragraph*{\textbf{Example Protocol}}
As an example, consider the two-phase commitment protocol.
The ideal functionality of this protocol consists of a \emph{sender} $S$
and \emph{receiver} $R$ connected to a trusted third-party, which we
name $\fcomm$.
The protocol initiates with $S$ sending a $\mb{commit}$ message to $\fcomm$
indicating its intent to \emph{commit} to a bit.
Next, $S$ sends this committed bit to $\fcomm$.
After receiving the committed bit, $\fcomm$ sends a $\mb{commit}$ message
to $R$ indicating that a bit has been committed to, but does not reveal
this bit to $R$.
At a later time, $S$ sends an $\mb{open}$ message to $\fcomm$ expressing
that $S$ wishes to reveal the secret bit to $R$.
Receiving this message, $\fcomm$ in turn sends an $\mb{open}$ message
to $R$ followed by this bit.
The protocol concludes with each party (process) terminating.

In the session-typed setting, we use typed channels to connect two
parties. For instance, the channel connecting $S$ to $\fcomm$
has the session type $\m{sender}$ defined as
\[
  \mi{stype} \; \m{sender} = \ichoice{\mb{commit} : \m{bit} \product
  \ichoice{\mb{open} : \one}}
\]
The type constructor $\ichoice$ denotes an \emph{internal choice}
(here with only one choice) dictating that $S$ must send the
$\mb{commit}$ message to $\fcomm$.
Next, we use the type constructor $\product$ to denote that $S$
sends a value of type $\m{bit}$ ($\m{bit} \product \ldots$).
We then use the $\ichoiceop$ constructor again enforcing
that $S$ sends $\mb{open}$ to $\fcomm$.
Finally, the type $\one$ denotes termination, indicating that
$S$ will send $\m{close}$ message to $\fcomm$.

Analogously, the channel connecting $R$ and $\fcomm$ has
type $\m{receiver}$ defined as
\[
  \mi{stype} \; \m{receiver} = \echoice{\mb{commit} : 
  \echoice{\mb{open} : \m{bit} \arrow \one}}
\]
Type constructor $\echoice$ represents \emph{external choice}
which is the dual to internal choice.
It prescribes that $R$ must receive a $\mb{commit}$ message from $\fcomm$,
followed by an $\mb{open}$ message (using another $\echoiceop$ constructor).
$R$ must then receive a bit using the $\arrow$ constructor (dual to
$\product$).
Finally, the session terminates as indicated by type $\one$.

Protocols expressed via session types are strictly enforced by
process definitions.
As an illustration, consider the $\fcomm$ process that is connected
to both $S$ and $R$.
The process declaration is written as
\begin{lstlisting}[basicstyle=\small\ttfamily]
decl F_comm :
  (S : sender), (R : receiver) |- (F : 1)
\end{lstlisting}
Here, $\fcomm$ is the name of the process, and $S$ and $R$ are the names
of channels \emph{used} by $\fcomm$, while $F$ is the channel \emph{provided}
by $\fcomm$.
Every session-typed process provides a unique channel while acting as a client
of a non-negative number of channels.
The used channels with their types are written to the left of the turnstile
($\vdash$) while the offered channel and type are written on the right.
This is analogous to function definitions where used channels correspond to
arguments, while offered channel corresponds to the result.

The $\fcomm$ process is defined as follows:
\begin{lstlisting}[basicstyle=\small\ttfamily, numbers=left,xleftmargin=2em]
proc F <- F_comm S R =
  case S (
    commit => b = recv S ;
              R.commit ;
              case S (
                open => R.open ;
                        send R b ;
                        wait S ; wait R ;
                        close F ) )
\end{lstlisting}
The process first case analyzes on channel $S$ branching on the
message received.
Since there is only one choice $\mb{commit}$, we only have one
branch in the definition.
$\fcomm$ then receives the bit $b$ (line 3) on $S$, followed by sending the
commit message on channel $R$ (line 4).
Once $\fcomm$ receives the $\mb{open}$ message on $S$, it sends the
$\mb{open}$ message on $R$ (line 6), followed by the bit $b$ (line 7).
Finally, the process waits for channels $S$ and $R$ to close (line 8),
followed by ultimately closing the channel $F$ (line 9).

\begin{figure*}[!ht]
In this section we use our emulation definition to model a simple commitment protocol in the random oracle mode.
We present a complete emulation from the canonical $\F_{\msf{com}}$ definition with a simulator for the dummy adversary.

The ideal functionality \Fcom is the same one introduced in Figure~\ref{fig:fcom} in Section~\ref{sec:nomosuc}. 
In FIgure~\ref{lst:fcom} we present the Nomos definition of the same functionality and provide the channel types in FIgure~\ref{fig:fcomtypes}.
We elide some of the clutter of acquiring and releasing shared channels in the form of: \texttt{\$p2f $\leftarrow$ acquire \#p\_to\_f} for clarity. 
Wherever a linear channel like \texttt{\$p2f} is used it is in fact an acquired shared channel \texttt{\#p\_to\_f}.

The key difference to note between \Fcom and its Nomos version is that the functionality is split up into two processes rather than compressed into one.
This design decision is required because of how Nomos cycles between processes in a round-robin fashion and the communicator design.
Therefore, processes must recurse when there is no message to be read and move to the next processes after the first expected message is received--in this case the \msf{P2FCommit(b)} message.

\begin{figure}
\centering
\msf{type} \msf{Ip2f} = \msf{P2FCommit} of \msf{Bit} | \msf{P2FOpen}

\msf{type} \msf{If2p} = \msf{F2PCommit} | \msf{F2POpen} of \msf{Bit}

\msf{type} \msf{Ip2f} = \msf{SCommit} of Bit | \msf{SOpen}

\msf{type} \msf{If2p} = \msf{RCommit} | \msf{ROpen} of Bit

\msf{type} \msf{Rp2f} = \msf{SHash} of \msf{Int} | \msf{Send} of \msf{pid} \textasciicircum \msf{pid} \textasciicircum \msf{Int}

\msf{type} \msf{Rf2p} = \msf{Pre} of \msf{Int} | \msf{RHash} of \msf{Int} | \msf{MSG} of \msf{pid} \textasciicircum \msf{pid} \textasciicircum \msf{Int}

\caption{Types for the channels in the ideal world for \Fcom. Notice that Ip2f and If2p is the type of the channels \msf{z2p} and \msf{p2z} as they much match for both worlds and the ideal world parties simply forward messages to the functionality. The \msf{p2f} and \msf{f2p} channels are specific to the real and ideal world as the functionalities are not the same. Hence the real-world \msf{p2f} is typed with \msf{Rp2f} for the random oracle and the ideal world \msf{p2f} is typed with \msf{Ip2f} for \Fcom.}
\label{fig:fcomtypes}
\end{figure}

\begin{figure*}
\begin{lstlisting}[basicstyle=\small\BeraMonottFamily]
proc F_code:
  (s: sid), (k: Int), (rng: [Bit]), (clist: list[Int]),
  (#p_to_f: comm[pid ^ Ip2f]{Ip2fn}), (#f_to_p: comm[pid ^ If2p]{If2pn}),
  (#a_to_f: comm[Ia2f]{Ia2fn}), (#f_to_a: comm[If2a]{If2an})  |- ($ch: FtOE) =
{
  case $p2f (
    yes =>	
      pid, msg = recv $p2f ;
      get $pwf {Ip2fn} ;
      case msg (
        P2FCommit(b) =>	
          if pid == 1
          then
            send F2PCommit $f2p ;
            pay {If2pn} $f2p ;
            $ch <- F_com_open s k rng clist #p_to_f #f_to_p b ;
          end
      )
   | no =>  
       $ch <- F_code s k rng clist #p_to_f #f_to_p ;
  )
}

proc F_code_open:
{
  case $p2f (
    yes =>	
      pid, msg = recv $p2f ;
      get $p2f {0} ;
      case msg (
        P2FOpen =>	
          if pid == 1
          then
            send F2POpen(b) $f2p ;
            pay {0} $f2p ;
            $ch <- 1 ;
          end
      )
   | no =>
       $ch <- F_code_open s k rng clist #p_to_f #f_to_p b ;
  )
}
\end{lstlisting}
\end{figure*}

The real world protocol for commitment follows a simple communication patter:
\begin{enumerate}
\item On input bit $(\msf{P2FCommit}\ b)$ from the environment, the committer queries the random oracle with the message $(\msf{SHash}\ b | r)$ where $r \xleftarrow{\$} \{0,1\}^k$.
The returned ``hash value'' is sent to the receiver as the commitment.
\item On input \msf{P2FOpen} from \Environment, the committer sends $(\msf{Send}\ p_L\ b\ r)$ to the receiver.
\item The receiver checks that the commitment is correct be querying \Fro in the same way and asserting that the has returned $(\msf{RHash}\ h)$ is the same as the one sent by $p_C$.
\item \todo{the type of Rf2p is kind of wrong so need to correct it}
\end{enumerate}

We provide only a simulator for the dummy adversary as that guarantees a simulator for all adversaries required our emulation definition.
The simulator for commitment is relatively simple so we only provide a high-level description here and leave the full simulator code to the appendix.
The simulator internally simulats the random oracle by maintaining a table of key-value pairs that it can control entirely.
If the committer is corrupt:
\begin{enumerate}
\item The simulator can not determine the bit \Environment wants to commit to so selects a random bit when activatd by the environment and gives it as input to the corrupted committer.
\item When it's asked to open the commitment it simply forwards the request to the corrupt committer and stops.
\end{enumerate}
If the receiver is corrupt:
\begin{enumerate}
\item When activated by the receiver with (\msf{F2PCommit}), the simulator generates some random string $h$ to represent the commitment, stores it, and sends ($\msf{P2A}\ \msf{MSG}(p_C, p_R, h)$) to \Environment.
\item When it receives ($\msf{F2POpen}\ b$) from the receiver, it returns $(\msf{P2A}\ \msf{MSG}(p_C, p_R, b, r)$ to \Environment where $r$ is a randomly generated sequence keeping the pair $(b | r, h)$ as the corresponding entry in the table.
\item When activated by \Environment to check the commitment, \Simulator simply returns the commitment hash or creates a new one.
\end{enumerate}

\paragraph{Simulator Well-Matched}
It is immediately obvious that the constructed simulator is well-typed if the dummy adversary is well-typed with the given type parameters.
The simulator receives 1 import token per activation from \Environment which suffices to simulated \Fro internally. 
Subsequently, \Simulator keeps all of the import it receives, and, therefore when one of the partiesis corrupt a simple bounding polynomial can be given as:
\[
	T_{\Dummysim}(n) = T_{\Fro}(n) + O(1)
\]
where $T_{\Fro}$ is a satisfying polynomial for \Fro. The additional constant factor simply accounts for sending messages to the corrupt parties.
Therefore,
\begin{gather}
	\forall \Environment, \langle \Environment \leftrightarrow \DummyAdv \rangle \Rightarrow \langle \Environment \leftrightarrow \Dummysim \rangle
\end{gather}


\begin{figure*}
\begin{lstlisting}[basicstyle=\BeraMonottFamily]
(* Z2P interface *)
type Ip2f = P2FCommit of Bit | P2FOpen ;
type If2p = F2PCommit | F2POpen of Bit ;

(* Ideal World *)
type Ip2f = SCommit of Bit | SOpen ;
type If2p = RCommit | ROpen of Bit ;

type Ia2f = 0 ;
type If2a = 0 ;

type Ia2p = Ip2f ;	(* crupt input is same as z2p *)
type Ip2a = If2p ;

(* Real World *)
type Rp2f = SHash of Int | Send of pid ^ pid ^ Int ;
type Rf2p = Pre of Int | RHash of Int | MSG of pid ^ pid ^ Int ;

type Ra2f = A2Hash of Int ;
type Rf2a = Hash2A of Int ;

type Ra2p = Rp2f ;
type Rp2a = Rf2p ;

(* the import here is given as those for the dummy adversary in the real world *)
p2zn <- 0 ; z2pn <- 1 ;
a2zn <- 0 ; z2an <- 1 ; 

Rf2pn <- 0 ; Rp2fn <- 1 ;
Rp2an <- 0 ; Ra2pn <- 1 ;
Rf2an <- 0 ; Ra2fn <- 1 ;

If2pn <- 0 ; Ip2fn <- 0 ;
Ip2an <- 0 ; Ia2pn <- 0 ;
I

(* channels *)
#z_to_p <- comm[pid ^ Ip2f]
#p_to_z <- comm[pid ^ If2p]
#z_to_a <- comm[ z2d[Ra2p][Ra2f] ] ;
#z_to_z <- comm[ d2z[Rp2a][Rf2a] ] ;


(* Real World exec PI *)
execUC[Ip2f][If2p][Rp2f][Rf2p][Rp2a][Ra2p][Rf2a][Ra2f][a2z][z2a]
	  {p2zn}{z2pn}{f2pn}{p2fn}{p2an}{a2pn}{f2an}{a2fn}{a2zn}{z2an}

(* Ideal world exec PHI *)
execUC[If2p][Ip2f][Ip2f][If2p][Ip2a][Ia2p][If2a][Ia2f][a2z][z2a]
	  {p2zn}{z2pn}






\end{lstlisting}
\end{figure*}

\caption{The $\mathcal{F}_{\msf{comm}}$ commitment ideal functionality in Nomos. The types for the sender and receiver channel define what inputs they can give to the functionality and what messsages are sent from the functionality back to the receiver.}
\label{fig:nomos:commitment}
\end{figure*}

\subsection{Formal Description of the NomosUC Language}

The core calculus of the NomosUC language is based on
\emph{session types}: a type discipline for communication-centric programming
based on message passing via channels. Session-typed channels describe and
enforce the protocol of communication among processes. The base system of
session types is derived from a Curry-Howard interpretation~\cite{CairesCONCUR2010}
of intuitionistic linear logic~\cite{GirardTCS1987}.
As a result, purely linear
propositions can be viewed as resources that must be used \emph{exactly
once} in a proof yielding the sequent $A_1, \ldots, A_n \vdash C$
where $A_1, \ldots, A_n$ are linear antecedents, while $C$ is the linear
succedent. Under this correspondence, a process term $P$ is assigned to
the above judgment and each hypothesis as well as the conclusion is
labeled with a \emph{channel}:
\[
x_1 : A_1, \ldots, x_n : A_n \vdash P :: (z : C)
\]
The resulting judgment states that process $P$ \emph{provides} a service
of session type $A$ along channel $z$, \emph{using} the services of session
types $A_1, \ldots, A_n$ provided along channels $x_1, \ldots, x_n$
respectively. We mandate all channel names to be distinct for the judgment
to be \emph{well-formed}. The antecedents are often abbreviated to $\D$.

The operational semantics for session-typed programs are formalized as a
system of \emph{multiset rewriting rules}~\cite{Cervesato2009SSOS}.
We introduce semantic objects $\proc{c}{P}$ and $\msg{c}{M}$ describing
process $P$ (or message $M$) providing service along channel $c$.
Remarkably, in this formulation, a message is just a particular form of process,
thereby not requiring any special rules for typing; it can be typed just as processes.

Formally, the typing judgment for processes in NomosUC is written as
$\Tokens \semi \Psi \semi \D \entailpot{q} P :: (x : A)$.
Here, $\Psi$ denotes the functional data structures and $\D$ collects the
session-typed channels along with an optional write token $\wt$
(to resolve non-determinism in the semantics).
The process is denoted by $P$ that offers channel $x$ of type $A$.
In addition, the annotation $q$ denotes a natural number denoting the
total potential stored in process $P$.
Finally, $\Tokens$ contains the type and quantity of import tokens
stored in the process.
We also extend the semantic objects to $\proc{c}{w, P}$ and $\msg{c}{w, P}$
where work counter $w$ stores the work performed by process $P$ (resp.
message $M$).
We will gradually explain each component of the language, initiating
with the basic system of session types.
For simplicity of exposition, we will display the yet unexplained
parts of the system in blue.

The Curry-Howard correspondence gives each linear logic connective an
interpretation as a session type.
In this article, we restrict to a subset of these connectives that
are sufficient for our language and purposes.
We follow a detailed description of each of these session type constructors.

\paragraph*{\textbf{Internal Choice}}
The internal choice $\ichoice{\ell : A_\ell}_{\ell \in L}$ constructor
is an $n$-ary labeled generalization of the additive disjunction $A \oplus B$.
A process that provides $x : \ichoice{\ell : A_\ell}_{\ell \in L}$ can send
any label $k \in L$ along $x$ and then continue by providing $x : A_k$. The
corresponding process is written as $(\esendl{x}{k} \semi P)$, where
$P$ is the continuation that provides $A_k$. This typing is formalized
by the \emph{right rule} $\oplus R$ in linear sequent calculus. The
corresponding client branches on the label received along $x$ as specified
by the \emph{left rule} $\oplus L$.
\begin{mathpar}
  \footnotesize
  \infer[{\oplus}R]
  {\B{\Tokens \semi \Psi} \semi \wt, \D \entailpot{\B{q}} (\esendl{x}{k} \semi P) ::
    (x : \ichoice{\ell : A_\ell}_{\ell \in L})}
  {(k \in L) \qquad \B{\Tokens \semi \Psi} \semi \D \entailpot{\B{q}} P :: (x : A_k)}
\end{mathpar}
\begin{mathpar}
  \footnotesize
  \infer[{\oplus}L]
  {\B{\Tokens \semi \Psi} \semi \D, (x : \ichoice{\ell : A_\ell}_{\ell \in L})
    \entailpot{\B{q}} \ecase{x}{\ell}{Q_\ell}_{\ell \in L} :: (z : C)}
  {(\forall \ell \in L) \qquad \B{\Tokens \semi \Psi} \semi \wt, \D, (x : A_\ell)
    \entailpot{\B{q}} Q_\ell :: (z : C)}
\end{mathpar}
Additionally, the provider should possess the write token to be able to send the
label $k$. Dually, the client receives the write token with the label to continue
execution.

Operationally, since communication is asynchronous, the process
$(\esendl{c}{k} \semi P)$ sends a message $k$
along $c$ and continues as $P$ without waiting for it to be received.
As a technical device to ensure that consecutive messages on a
channel arrive in order, the sender also creates a fresh continuation
channel $c'$ so that the message $k$ is actually represented as
$(\esendl{c}{k} \semi \fwd{c}{c'})$ (read: send $k$ along $c$ and
continue as $c'$). The provider substitutes $c'$ for $c$ enforcing
that the next message is sent on $c'$.
The work counter of the process remains unaltered, and the new message
is created with work $0$.
\begin{tabbing}
$(\oplus S) : \proc{c}{w, \esendl{c}{k} \semi P} \step$ \= $\proc{c'}{w, [c'/c]P},$\\
\> $\msg{c}{0, \esendl{c}{k} \semi \fwd{c}{c'}}$
\end{tabbing}
When the message $k$ is received along $c$, the client selects branch
$k$ and also substitutes the continuation channel $c'$ for $c$, thereby
ensuring that it receives the next message on $c'$. This implicit
substitution of the continuation channel ensures the ordering of the
messages.
The client process also collects the work performed by the message.
\begin{tabbing}
$(\oplus C) :$ \= $\msg{c}{w, \esendl{c}{k} \semi \fwd{c}{c'}},
\proc{d}{w', \ecase{c}{\ell}{Q_\ell}}$\\
\> $\qquad \step \proc{d}{w+w',[c'/c]Q_k}$
\end{tabbing}

\paragraph*{\textbf{External Choice}}
The dual of internal choice is \emph{external choice} $\echoice{\ell :
A_\ell}_{\ell \in L}$, the $n$-ary labeled generalization of the
additive conjunction $A \with B$. This dual operator simply reverses
the role of the provider and client. The provider process of
$x : \echoice{\ell : A_\ell}_{\ell \in L}$ branches on receiving a label
$k \in L$ (described in $\with R$), while the client sends this label
(described in $\with L$).
\begin{mathpar}
  \footnotesize
  \infer[\with R]
  {\B{\Tokens \semi \Psi} \semi \D \entailpot{\B{q}} \ecase{x}{\ell}{P_\ell}_{\ell \in L} ::
    (x : \echoice{\ell : A_\ell}_{\ell \in L})}
  {(\forall \ell \in L) \qquad \B{\Tokens \semi \Psi} \semi \wt, \D
    \entailpot{\B{q}} P_\ell :: (x : A_\ell)}
\end{mathpar}
\begin{mathpar}
  \footnotesize
  \infer[\with L]
  {\B{\Tokens \semi \Psi} \semi \wt, \D, (x : \echoice{\ell : A_\ell}_{\ell \in L})
    \entailpot{\B{q}} \esendl{x}{k} \semi Q :: (z : C)}
  {\B{\Tokens \semi \Psi} \semi \D, (x : A_k) \entailpot{\B{q}} Q :: (z : C)}
\end{mathpar}
Dual to internal choice, the client contains the write token which is
sent to the provider along with the label.
The operational semantics rules are just the inverse of internal choice,
and therefore skipped for brevity.

\paragraph*{\textbf{Termination}}
The type $\one$, the multiplicative unit of linear logic, represents
termination of a process, which (due to linearity) is not allowed to use
any channels. A terminating process offering on $x : \one$ simply
closes channel $x$ while the client waits for this close message to arrive.
\begin{mathpar}
  \footnotesize
  \infer[{\one}R]
  {\B{\Tokens \semi \Psi} \semi \wt \entailpot{\B{q}} \eclose{x} :: (x : \one)}
  {\B{q = 0}}
  \and
  \infer[{\one}L]
  {\B{\Tokens \semi \Psi} \semi \D, (x : \one) \entailpot{\B{q}} (\ewait{x} \semi Q) :: (z : C)}
  {\B{\Tokens \semi \Psi} \semi \wt, \D \entailpot{\B{q}} Q :: (z : C)}
\end{mathpar}
Similar to internal choice, the closing process transfers the write
token to its waiting client along with the close message.
Additionally, the terminating process does not store
any potential since it cannot take any further execution steps.
% Operationally, the provider converts into a closing message
% with no continuation since the offered channel terminates.
% \begin{tabbing}
% $(\one S) : \proc{c}{\eclose{c}} \step \msg{c}{\eclose{c}}$ \\
% $(\one C) : \msg{c}{\eclose{c}}, \proc{d}{\ewait{c} \semi Q} \step
% \proc{d}{Q}$
% \end{tabbing}

% The provider receives the branching label $k$ sent by the provider. Both
% processes perform appropriate substitutions to ensure the order of messages
% sent and received is preserved.
% \[
% \begin{array}{lll}
% (\with S) & \proc{d}{\esendl{c}{k} \semi Q} \step \msg{c'}{\esendl{c}{k}
% \semi \fwd{c'}{c}}, \proc{d}{[c'/c]Q} & \fresh{c'} \\
% (\with C) & \proc{c}{\ecase{c}{\ell}{Q_\ell}_{\ell \in L}},
% \msg{c'}{\esendl{c}{k} \semi \fwd{c'}{c}} \step \proc{c'}{[c'/c]Q_k}
% \end{array}
% \]

\paragraph*{\textbf{Exchanging Functional Data}}
Communicating a \emph{value} of the functional fragment along a channel
is expressed at the type level by adding the following two session types.
\begin{center}
\begin{minipage}{0cm}
\begin{tabbing}
$A ::= \ldots \mid \tau \arrow A \mid \tau \product A$
\end{tabbing}
\end{minipage}
\end{center}
Here, $\tau$ describes a functional type, e.g. $\m{int}, \m{bool}, \tau \; \m{list}$, etc
(we assume the language contains standard functional types).
The type $\tau \arrow A$ prescribes receiving a value of type $\tau$
with continuation type $A$, while its dual $\tau \product A$ prescribes
sending a value of type $\tau$ with continuation $A$. The corresponding
typing rules for arrow ($\arrow R, \arrow L$) are given
below (rules for $\product$ are inverse).
\begin{mathpar}
  \footnotesize
  \infer[\arrow R]
  {\B{\Tokens} \semi \Psi \semi \D \entailpot{\B{q}}
  \erecvch{x}{v} \semi P :: (x : \tau \arrow A)}
  {\B{\Tokens} \semi \Psi, (v : \tau) \semi \wt, \D \entailpot{\B{q}}
  P :: (x : A)}
  %
  \and
  %
  \inferrule*[right = $\arrow L$]
  {\B{r = p+q} \qquad
  \B{\Psi \share (\Psi_1, \Psi_2)} \qquad
  \Psi_1 \exppot{\B{p}} M : \tau \\
  \B{\Tokens} \semi \Psi_2 \semi \D, (x : A) \entailpot{\B{q}}
  Q :: (z_k : C)}
  {\B{\Tokens} \semi \Psi \semi \wt, \D, (x : \tau \arrow A)
  \entailpot{\B{r}} \esendch{x}{M} \semi Q :: (z : C)}
\end{mathpar}
As indicated in the $\arrow R$ rule, receiving a value $y : \tau$ on a channel
$x : \tau \arrow A$ adds it to the functional context $\Psi$. On the
other hand, sending (value of) expression $M$ on channel $x : \tau \arrow A$
requires that $M$ has type $\tau$ (third premise).
The premises indicated in blue describe how potential is divided across
the functional and session-typed layers and will be described next.
Intuitively, the potential in functional context $\Psi$ is \emph{shared}
between $\Psi_1$ and $\Psi_2$ (second premise); $\Psi_1$ is used to type
$M$ while $\Psi_2$ is passed on to the continuation $Q$.

\paragraph*{\textbf{Import Tokens}}
NomosUC differs from the UC framework in one important aspect: UC employs
import tokens to bound the runtime of ITMs.
On the other hand, NomosUC simplifies matters by relying on the notion of
abstract potential and requiring that every execution step costs $1$ unit
of potential.
As a consequence, however, we need to connect import tokens to potential.
To this end, we introduce a novel token context $\Tokens$ in the process typing
judgment.
This context stores the number and type of each token and is written as
$\Tokens ::= K \hookrightarrow n, \; \Tokens \mid \cdot$ denoting that the
process stores $n$ tokens of type $K$.
We also introduce a novel construct $\m{generatePotential} \; f \; (m : K)$
which consumes $m$ units of import tokens of type $K$ to generate $f(m)$
units of potential. The corresponding typing rule is
\begin{mathpar}
  \inferrule*[right=$\m{pot}$]
  {\Tokens, K \hookrightarrow n \semi \Psi \semi \wt, \D \entailpot{q+f(m)} P :: (x : A)}
  {\Tokens, K \hookrightarrow (n+m) \semi \Psi \semi \wt, \D \entailpot{q} \hspace{8em} \\
    \hspace{5em}\m{generatePotential} \; f \; (m : K) \semi P :: (x : A)}
\end{mathpar}
The process initiates with potential $q$ and ends up with $q+f(m)$ potential
by deducting $m$ import units of type $K$ from $\Tokens$.
We also introduce a novel construct to create a new import token type
$\m{convertToken} \; f \; K' \; (m : K)$.
\begin{mathpar}
  \inferrule*[right=$\m{tok}$]
  {\Tokens, K \hookrightarrow n, K' \hookrightarrow f(m) \semi
  \Psi \semi\wt, \D \entailpot{q} P :: (x : A)}
  {\Tokens, K \hookrightarrow (n+m) \semi \Psi \semi \wt, \D \entailpot{q} \hspace{8em} \\
    \hspace{5em}\m{convertToken} \; f \; K' \; (m : K) \semi P :: (x : A)}
\end{mathpar}
The above construct deducts $m$ import of type $K$ and produces $f(m)$ import
of type $K'$ as exemplified by the token context.


\paragraph*{\textbf{Exchanging Potential}}
To share and amortize execution cost, NomosUC allows exchange of potential
between processes.
To transfer potential, NomosUC employs two new type constructors.
\begin{center}
\begin{minipage}{0cm}
\begin{tabbing}
$A ::= \ldots \mid \tpaypot{A}{r} \mid \tgetpot{A}{r}$
\end{tabbing}
\end{minipage}
\end{center}
The provider of $x : \tgetpot{A}{r}$ is required to receive
$r$ units of potential from the client using the construct
$\eget{x}{r}$. Dually, the client needs to pay this potential
using the construct $\epay{x}{r}$.
The corresponding typing rules are
\begin{mathpar}
  \footnotesize
  \infer[\getpot R]
  {\B{\Tokens} \semi \Psi \semi \D \entailpot{q} \eget{x}{r} \semi P ::
  (x : \tgetpot{A}{r})}
  {p = q+r \qquad
  \B{\Tokens} \semi \Psi \semi \wt, \D \entailpot{p} P :: (x : A)}
  %
  \and
  %
  \infer[\getpot L]
  {\B{\Tokens} \semi \Psi \semi \wt, \D, (x : \tgetpot{A}{r}) \entailpot{q}
  \epay{x}{r} \semi P :: (z : C)}
  {q = p+r \qquad
  \B{\Tokens} \semi \Psi \semi \D, (x : A) \entailpot{p} P :: (z : C)}
\end{mathpar}
In the rule $\getpot R$, process $P$ storing
potential $q$ receives $r$ units, thus the continuation
executes with $p = q+r$ units of potential. 
%
In the dual rule $\getpot L$, a process storing potential $q = p+r$
sends $r$ units along channel $x$ leaving $p$ units with
the continuation.
The typing rules for the dual constructor $\tpaypot{A}{r}$
are the exact inverse.
Similar to prior rules, the sender transfers the write token $\wt$
along with the potential to the receiver.

The purpose of introducing potential into NomosUC is to bound the
number of execution steps.
Therefore, we introduce the $\etick{r}$ construct that consumes $r$
potential from the stored process potential $q$, and the continuation remains with
$p = q-r$ units, as described in the rule below.
\begin{mathpar}
  \footnotesize
  \infer[\m{tick}]
  {\B{\Tokens} \semi \Psi \semi \wt, \D \entailpot{q} \etick{r} \semi P :: (x : A)}
  {q = p + r \qquad
  \B{\Tokens} \semi \Psi \semi \wt, \D \entailpot{p} P :: (x : A)}
\end{mathpar}
NomosUC is equipped with a cost instrumentation engine that automatically
inserts a $\etick{1}$ construct before each primitive operation.
This enables us to simulate the cost model that counts the total number of
execution steps.
However, since ticks are not tied directly to the type system, the programmer
can modify the cost model to only count the resource they are interested in
(e.g., message exchange, process spawns, etc.).

% \begin{mathpar}
%   \D_1 \equiv_Z \D_2 \\
%   \D \overset{(import, potential, cost)}{\vDash} P :: \D' \\
%   A \equiv B \\
%   \infer[]
%   {\vars \vdash \D_1, (x : A) \equiv \D_2, (x : B)}
%   {\vars \vdash \D_1 \equiv \D_2 \and \vars \vdash A \equiv B}
% \end{mathpar}

\paragraph*{\textbf{Shared Channels}}
Until now, we have only described the linear fragment of session types
in Nomos.
Unfortunately, this fragment imposes a strong restriction on programs.
The only provision to spawn new processes is when a parent process creates a new
child process, and uses an exclusive linear channel to communicate with the child.
Thus, any two processes connected by a channel inherently maintain this parent-child
relationship.
Intuitively, this leads to a linear tree-like hierarchy among the processes,
thus preventing a cycle in the process graph.

Unfortunately, this restriction precludes practical programming scenarios
where process topologies indeed have a cyclic dependency (e.g. ring networks,
dining philosophers, etc.).
Recognizing this limitation, Balzer et al.~\cite{Balzer17icfp} proposed
a \emph{shared} extension of session types that allows arbitrary process topologies.
The types are extended as follows:
\begin{center}
\begin{minipage}{0cm}
\begin{tabbing}
$A_L ::= \down A_S \mid \ldots \text{(all linear types $A$ so far)}\ldots$\\
$A_S ::= \up A_L$
\end{tabbing}
\end{minipage}
\end{center}
We have found this extension exceedingly helpful in the design and implementation
of cryptographic protocols.

Shared session types impose an \emph{acquire-release} discipline on processes; 
a client must acquire the channel offered by a shared process to interact with it
and must release this channel after the interaction.
The corresponding typing rules are
\begin{mathpar}
  \footnotesize
  \infer[\up L]
  {\B{\Tokens} \semi \Psi \semi \wt, \D, (x : \up A_L)
  \entailpot{q} \eacquire{y}{x} \semi Q :: (z : C)}
  {\B{\Tokens} \semi \Psi \semi \D, (y : A_L)
  \entailpot{q} Q :: (z : C)}
  %
  \and
  %
  \infer[\up R]
  {\B{\Tokens} \semi \Psi \semi \D \entailpot{q}
  \eaccept{y}{x} \semi P :: (x : \up A_L)}
  {\B{\Tokens} \semi \Psi \semi \wt, \D \entailpot{q} P :: (y : A_L)}
\end{mathpar}
The $\up L$ rule describes a client acquiring a shared channel $x$
and obtaining a private linear channel $y$ along which it can communicate
with the corresponding acquired process.
Correspondingly, the $\up R$ rule describes the shared process
accepting the acquire request and creating the fresh linear channel $y$.
The release-detach rules corresponding to the $\down$ type constructor
are exact dual of acquire-accept.

An important caveat here is that shared channels can introduce non-determinism
in the semantics.
The only source of non-determinism is that a shared process can latch on to
any of the acquiring clients.
To address this problem, we require that the acquiring client \emph{possess
the write token}.
Since write tokens are treated as a linear quantity, only one of the client can
possess it enabling only that process to acquire the shared channel.
Remarkably, this write token can resolve both read and write non-determinism
due to linearity of the channels.


\paragraph*{\textbf{Process Definitions}}
Process definitions have the form
$\Psi \semi \D \entailpot{q} f\{\Tokens\} = P :: (x : A)$ where $f$
is the name of the process and $P$ its definition.
We parameterize the process $f$ with the number and type of
tokens it would need.
All definitions are collected in a fixed global signature $\Sg$.
Also, since process definitions are mutually recursive, it is required that
for every process in the signature is well-typed w.r.t. $\Sg$.
A new instance of a defined process $f$ can be spawned with
the expression $\procdef{f\{\Tokens\}}{\overline{y}}{x} \semi Q$
where $\overline{y}$ is a sequence of variables matching the
antecedents $\Psi$ and $\D$.
Sometimes a process invocation is a \emph{tail call}, written without
a continuation as $\procdef{f\{\Tokens\}}{\overline{y}}{x}$.
This is a short-hand for
$\procdef{f\{\Tokens\}}{\overline{y}}{x'} \semi \fwd{x}{x'}$ for a
fresh variable $x'$, that is, a fresh channel is created and
immediately identified with $x$.

\subsection{Preservation and Progress}
The main theorems that exhibit the deep connection between our type
system and the operational semantics are the usual \emph{type
  preservation} and \emph{progress}, sometimes called \emph{session
  fidelity} and \emph{deadlock freedom}, respectively.

So far, I have only described individual processes. However, processes
exist in a \emph{configuration}. A process configuration is a multiset
of semantic objects, $\proc{c}{P}$ and $\msg{c}{M}$, where any
two offered channels are distinct. A key question is how to type these
configurations. Since they consist of both processes and messages, they
both \emph{use} and \emph{provide} a collection of channels.
And even though a configuration is treated as a multiset, typing imposes
a partial order on the processes and messages where a provider of a
channel appears to the left of its client.

A configuration is typed w.r.t. a signature providing the type declaration
of each process.
A signature $\Sg$ is \emph{well formed} if
(a) every type definition $V = A_V$ is \emph{contractive},
and (b) every process definition
$\D \vdash f = P :: (x : A)$ in $\Sg$
is well typed according to the process typing judgment, i.e.
$\Sg \semi \D \vdash P :: (x : A)$.

I use the following judgment to type a configuration.
\[
\Sg \semi \D_1 \vDash \config :: \D_2
\]
It states that $\Sg$ is well-formed
and that the configuration $\config$
uses the channels in the context $\D_1$ and provides
the channels in the context $\D_2$.
\begin{figure}[t]
\begin{mathpar}
\infer[\m{empty}]
{\Sg \semi \D \vDash (\cdot) :: \D}
{}
\and
\infer[\m{compose}]
{\Sg \semi \D_0 \vDash (\config_1 \; \config_2) :: \D_2}
{\Sg \semi \D_0 \vDash \config_1 :: \D_1 \qquad
\Sg \semi \D_1 \vDash \config_2 :: \D_2}
\and
\infer[\m{proc}]
{\Sg \semi \D, \D_1 \vDash \proc{c}{P} :: (\D, (c : A) )}
{\Sg \semi \D_1 \vdash P :: (c : A)}
\and
\infer[\m{msg}]
{\Sg \semi \D, \D_1 \vDash \msg{c}{P} :: (\D, (c : A) )}
{\Sg \semi \D_1 \vDash P :: (c : A)}
\end{mathpar}
\caption{Typing rules for a configuration}
\label{fig:config_typing}
\end{figure}
The configuration typing judgment is defined using
the rules presented in Figure~\ref{fig:config_typing}.
%
The rule $\m{empty}$ defines that an empty configuration
is well-typed. The rule $\m{compose}$
composes two
configurations $\config_1$ and $\config_2$; $\config_1$ provides
service on the channels in $\D_1$ while $\config_2$ uses
the channels in $\D_2$. The $\m{proc}$ rule creates a configuration
out of a single process. Similarly, the $\m{msg}$ rule creates a
configuration out of a single message.

\begin{theorem}[Type Preservation]
\label{thm:preservation}
If $\Sg \semi \D' \vDash \config :: \D$ and $\config \step \dc$,
then $\Sg \semi \D' \vDash \dc :: \D$.
\end{theorem}
\begin{proof}
  By case analysis on the transition rule, applying inversion to the
  given typing derivation, and then assembling a new derivation of
  $\dc$.
\end{proof}

A process or message is said to be \emph{poised} if it is trying to
communicate along the channel that it provides.  A poised process is
comparable to a value in a sequential language. A configuration is
poised if every process or message in the configuration is poised.
Conceptually, this implies that the configuration is trying to communicate
externally, i.e. along one of the channel it provides.
The progress theorem then shows that either a configuration can take a
step or it is poised.  To prove this I show first that the typing
derivation can be rearranged to go strictly from right to left and
then proceed by induction over this particular derivation.

\begin{theorem}[Global Progress]
\label{thm:progress}
\mbox{}
If $\cdot \vDash \config :: \D$ then either
\begin{enumerate}
\item[(i)] $\config \mapsto \dc$ for some $\dc$, or
\item[(ii)] $\config$ is poised.
\end{enumerate}
\end{theorem}
\begin{proof}
By induction on the right-to-left typing of $\config$ so that either
$\config$ is empty (and therefore poised) or
$\config = (\dc\; \proc{c}{P})$ or
$\config = (\dc\; \msg{c}{M})$. By induction hypothesis, $\dc$ can
either take a step (and then so can $\config$), or $\dc$ is poised.  In
the latter case, I
analyze the cases for $P$ and $M$, applying multiple steps of
inversion to show that in each
case either $\config$ can take a step or is poised.
\end{proof}


\subsection{UC Communicators}
One illustration of a use of shared session types is a \emph{communicator}.
We use communicators as message buffers between two arbitrary processes: a
\emph{sender} and a \emph{receiver}.
The communicator is connected to both the sender and the receiver using a shared
channel.
Intuitively, the communicator receives \emph{push} requests from the sender followed
by receiving a message and stores them internally.
Analogously, the communicator receives \emph{pop} requests from the receiver,
and responds appropriately with the message if one is stored inside the communicator.
Formally, a communicator has the following polymorphic session type
\begin{tabbing}
  $\mi{stype} \; \m{comm[msg]} =$\\
  \quad $\up \echoice{$\=$\mb{push} : \m{msg} \arrow
  \down \m{comm[msg]},$\\
  \>$\mb{pop} : \ichoice{$\=$\mb{yesmsg} : \m{msg} \product \down \m{comm[msg]},$\\
  \>\>$\mb{nomsg} : \down \m{comm[msg]} }}$
\end{tabbing}
The type $\m{comm}$ is parameterized by the type $\m{msg}$, i.e., the type of
messages in the buffer.
The type initiates with an $\up$ denoting that $\m{comm}$ is a shared session type.
The type prescribes that the communicator needs to be acquired by the sender (or receiver)
for further interaction.
Such an acquire-release discipline is automatically enforced by the shared session type.
Once acquired, the communicator can either receive $\mb{push}$ (from sender) or
$\mb{pop}$ requests (from receiver).
In the former case, the communicator receives a message of type $\m{msg}$, and
then detaches from the client using the dual $\down$ operator.
In the latter case, the communicator checks if it internally contains a message
for the receiver.
If yes, the communicator replies with the $\mb{yesmsg}$ label followed by sending
the message (the $\product$ constructor).
Otherwise, the communicator replies with the $\mb{nomsg}$ label.
In either case, the communicator then detaches from the client matching the $\down$
operator.
Internally, the communicator stores these messages in a first-in-first-out order.

The communicator is also the perfect opportunity to implement an unreliable
message buffer that can drop or reorder messages.
All we would need to do is change the internal implementation of the communicator
\emph{without} changing the offered session type.
