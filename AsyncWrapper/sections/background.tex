\subsection{Universal Composability}
The universal composability framework~\cite{uc} proposes a new framework for proving the security of cryptographic and distributed protocol.
Compared to previous works, the UC framework provides a stronger notion of security where protocols that are UC-secure are secure even when composed with arbitrary other protocols running concurrently. 

Such a strong notion of security is achieved through the real-ideal world paradigm.
The ideal world encompasses an ideal implementation of a protocol, called the \textit{ideal functionality} $\mathcal{F}$, which acts as a trusted third party that caputures all the desired security properties.
The ideal functionality is usually a simple definition making it trivial to prove its security properties.
The real world, on the other hand, consists of parties running an actual protocol, $\pi$, against a real adversary.

Security proofs in UC involve creating a simulator $\mathcal{S}$ in the ideal world that can simulate every potential attack on a real protocol in the real world.
If $\mathcal{S}$ can make the two worlds indistinguishable for any real world adversary $\mathcal{A}$ for all distinguishing environments $\mathcal{Z}$, then we say the protocol $\pi$ UC-emulates the ideal functionality $\mathcal{F}$.
Indistinguishability of the two worlds to any $\mathcal{Z}$ implies that the protocol $\pi$ must exhibit the same security properties as the ideal functionality $\mathcal{F}$ otherwise there should be sobe distinguishing environment. 
More formally, indistinguishability is stated:

$$ \text{EXEC}_{\mathcal{F},\mathcal{S},\Environment} \approx \text{EXEC}_{\pi,\mathcal{A},\Environment} $$

\paragraph{GUC-Framework}
A limitation of the standard UC framework is that it fails to capture scenarios where multiple protocol sessions share state through some ideal functionality.
The cannonical example used to insipire a more unconstrained environments is the signing of certificates. 
In the real world, a protocol party $p_i$ controlling some private key may sign messages that an adverasry could replay in other protocols $p_i$ takes part in.
The standard UC formulation, however, is incapable of modelling such an adversarial strategy as protocol sessions are completely distinct and can not share any state with each other---namely, an ideal functionality that stores signed certificates is bound only to the current protocol session.
\todo{Is is worth motivting the GUC framework rather than just introducing what it is and that the environment can also interact with a shared functionality?}


\subsection{The Import Mechanism}
A notion of resource-bound computation is necessary for the UC framework to reason about computationally efficient algorithms as well as the capabilities of ITIs under a particular resource constraint.
Often we would like to reason about adversarial capabilities under such constraints and perform efficient transformations (transforming an adversary into a simulator).

Current definitions of polynomial-time computation take the form of bounding the computation of an ITI by some polynomial $T$:
given an input of length $n$ a machine $\mu$ halts within $T(n)$ steps.
However, using the length of the inputs to the machine as $n$, in this case leads to an infinite runs problems identified by Canetti~\cite{uc}.
Machines that are locally $T(n)$-bounded are able to spawn other machines to the point that an infinite chain of such machines can be spawed where, although, each is locally $T$-bounded, the whole system of of ITMs is no longer PPT. 
Therefore, a new notion of $n$ is needed. 

The UC paper defines an import mechanism where the first ITI, the environment, is spawned with a polynomial amount of import which can be thought of as tokens or coins.
The environment can then activate other ITIs with some import tokens allowing them to run for as long as their import token balance permits. 
In this new definition, an ITI that is $T$-bounded takes at most $T(n')$ steps where $n'$ is the difference between the import it has received from incoming messages and outgoing import it's given to other machines.
Any ITI can write to any other ITI in it's communication set and send it import tokens until the balance of import is exhausted.
Eventually, the system of ITMs must halt when it runs out of import until it receives more from the environment.

\paragraph{Realizing Import in saUCy} \label{sec:realizeimport}
The import mechanism in the UC framework describes at a high level how to reason about resource bounded computation in the import model.
However, for our purposes it is necessary to instantiate it concretely and make explicit the conditions under which a machine halts as we care about an implementation the UC framework in our proposed Nomos language. 

In order to do this we must introduce a new concept called \textit{potential}. 
Potential is the basic unit of computation and machines must explicitly \textit{consume} their available import balance and convert it into usable potential through a bounding polynomial $T$.

Concretely, every ITM $\mu$ is spawned with a set of variables:
\begin{itemize}
\item $\msf{in}_t$: the total amount of import received as input by $\mu$.
\item $\msf{out}_t$: the total amount of import sent by $\mu$ to other machines.
\item $\msf{marked}_t$: the number of import tokens that have been consumed and used to generate potential through an input polynoial $T$.
\item $\msf{generate\_pot}(T, n)$: a function that marks $n$ available import tokens and creates $T(n)$ units of potential.
\end{itemize}

Over the lifetime of $\mu$ it will need to generate potential multiple times as it needs to perform more computation.
Eventually, thought, we can guarantee that every machine $\mu$ will \textit{halt} when it can no longer generate more potential, i.e.  $\msf{in}_t - \msf{out}_t - \msf{marked}_t > n$.
